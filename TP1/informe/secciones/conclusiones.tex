\section{Conclusiones}
%Odio las conclusiones cursis sobre que el TP está buenísimo, perdón
%El trabajo pr\'actico nos permiti\'o conocer de manera experimental c\'omo funciona un protocolo que conecta dos niveles: el nivel de enlace y el nivel de red. Pudimos observar la diferencia entre distintos tipos de redes utilizando solamente este protocolo. A su vez, pudimos aplicar de manera pr\'actica conceptos m\'as abstractos como el de la entrop\'ia y la informaci\'on, y hacer un an\'alisis de lo observado a partir de estos datos. Aprendimos tambi\'en la utilidad de herramientas como \emph{Scapy} y \emph{Wireshark}, que resultaron esenciales para hacer los an\'alisis mencionados.

Habiendo hecho la presentación de los resultados y sus comentarios específicos pertinentes, quedan para esta sección los comentarios generales sobre la experiencia realizada. Por un lado, vale destacar la posibilidad de conocer una red y su comportamiento observando únicamente mensajes de protocolos de capa de enlace, sin tener en cuenta los datos transmitidos por las capas superiores: con el análisis de los grafos de red vimos cómo se pueden diferenciar esquemas muy distintos (red privada vs. red pública), y dentro de las privadas (que por su compacidad y control nos permiten mayor seguridad a la hora de realizar afirmaciones) nodos y comportamientos particulares: nodos centralizadores (que podemos identificar con routers) y comportamientos anómalos (multiplicidad de solicitudes y ninguna respuesta, o una única solicitud y ninguna respuesta). Además, pudimos observar una situación muy poco esperable en el caso de la red pública, para la cual no podemos, en principio, establecer una respuesta.
%momento cursi
En cualquier caso, obliga a pensar que las redes públicas, por su importancia y alcance son manejadas con una complejidad mayor, y para comprender las cuales se necesitaría una herramienta (todavía) más poderosa.

Por otro lado no resulta menos interesante observar empíricamente las abstracciones propuestas por la teoría de la información, y cómo esta generaliza situaciones diversas, que también terminan por caracterizar a la red. Más allá de los valores de entropía obtenidos, sobre los cuales no pretendemos opinar cuantitativamente, y que podemos considerar próximos entre sí, resulta interesante los casos en que podemos diferenciar por tiempos, y nuevamente, caracterizar a las redes de las que provienen, teniendo en cuenta únicamente la interacción entre las computadoras y no el contenido o la razón de esta comunicación. Insistimos con el contraste entre la red ISP y la situación de oficina, ya que es un buen ejemplo de lo que se considera \emph{información} dentro de la teoría: no se trata (sólamente) de la presencia de mensajes, sino también de sus particularidades propias y generales: mientras que la primera debe su cantidad de información a la constante presencia de nodos diversos y una interacción moderada, la segunda es producto de una intensa interacción, donde parecería que se agotan las posibilidades, dejando a las horas siguientes repeticiones de un subconjunto de mensajes, hecho que, no sorprendentemente, no aporta grandes cantidades de información.

Para concluir, en el presente TP observamos distintas redes, utilizando por un lado un elemento puramente técnico (la relación entre direcciones IPs para deducir interacciones y distinguir participantes) y por otro, uno puramente teórico (adaptado para su cálculo efectivo) con el que podemos medir cuán relevante o no son las interacciones anteriormente detectadas, indicios suficientes para realizar caracterizaciones con sólo escuchar el mensajeo protocolar.
