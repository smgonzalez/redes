\begin{abstract}
\textbf{Resumen:} En el presente trabajo estudiaremos el comportamiento de una red a nivel de enlace, enfoc\'andonos en la relaci\'on entre esta capa y la capa superior, para lo cual analizaremos los paquetes ARP enviados a trav\'es de una red local. Adem\'as, vamos a escuchar pasivamente las redes seleccionadas, capturando paquetes ARP durante varias horas. Esta informaci\'on va a ser utilizada luego para ver, a trav\'es de grafos e histogramas, cu\'ales son los nodos significativos dentro de la misma. A su vez, utilizaremos los conceptos de entrop\'ia e informaci\'on para caracterizar las redes estudiadas, tomando los s\'imbolos de la fuente de alguna manera conveniente. Tanto para el env\'io de paquetes ARP, como para la captura de datos de la red, utilizaremos las herramientas \emph{Wireshark} y \emph{Scapy}. *** FALTA HABLAR DE RESULTADOS, O ESO HACIA EN METODOS.\\

\textbf{Palabras clave:} ARP, Teoría de la Información, Entropía, Scapy, Wireshark, Nivel de Enlace, IP, MAC, LAN.
\end{abstract}
