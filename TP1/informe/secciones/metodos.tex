\section{M\'etodos}

% LES COPIO LO QUE TENGO ANOTADO EN EL CUADERNO:
% Hablar de entropia (como la vamos a tomar, como vamos a sniffear. Tambien puede ponerse un pseudocodigo, o una explicacion del codigo

\subsection{Primera Consigna: Implementaci\'on de un cliente ARP}\label{sec:metodos_1}
El primer paso consiste en implementar una funci\'on que, dada una direcci\'on IP, env\'ie un mensaje ARP a trav\'es de la red local preguntando con qu\'e direcci\'on MAC se corresponde. Si recibe respuesta, se muestra por pantalla dicha direcci\'on; en caso contrario se indica que la IP es inexistente, o no puede ser alcanzada desde la red local.\\

Como mencionamos, esta consigna se implementa utilizando la herramienta \emph{scapy}. Ya que la misma funciona sobre python, programamos la funci\'on en este lenguaje. Para el env\'io del paquete ARP instanciamos un objeto del tipo ARP, y modificamos solamente el campo \emph{pdst}, ya que los otros tienen por defecto los valores que necesitamos (la IP de la m\'aquina que env\'ia, tipo de mensaje \emph{who-has}, etc). El env\'io en s\'i se realiza a trav\'es de la funci\'on \emph{sr} (send and receive), utilizando un timeout de 3 segundos. Esta funci\'on devuelve una lista cuyo primer elemento son los paquetes que respondieron, y el segundo los paquetes sin respuestas. Si el primer elemento no es vac\'io, se devuelven las direcciones MAC  recibidas; en caso contrario se indica que dicha direcci\'on no puede ser alcanzada en la red local. *** NOSOTROS DEVOLVEMOS UNA LISTA, PERO  CREO QUE SOLO DEVUELVE UNA MAC POR CADA WHO-HAS??\\

Utilizando la funci\'on programada, vamos a analizar qu\'e ocurre al suministrarle distintos tipos de direcciones IP. Todos los casos que mencionamos a continuaci\'on fueron fueron probados sobre una red wireless, cuya IP era 192.168.0.3. Los casos testeados fueron los siguientes:

\begin{enumerate}
 \item Direcciones que pertenecen a la red (para conocer las direcciones IP que se encuentran en la red utilizamos la funci\'on \emph{arping}, inclu\'ida dentro de \emph{scapy}): 192.168.0.1, 192.168.0.4, 192.168.0.6
 \item Direcciones con la m\'ascara de red correcta, pero que no pertenecen a hosts conectados a la misma. Por ejemplo: 192.168.0.2, 192.168.0.5, etc.
 \item Misma direcci\'on que la m\'aquina de origen (en este caso, 192.168.0.3).
 \item Direcci\'on IP correspondiente a la m\'aquina de origen, seg\'un se ve de afuera (utilizando por ejemplo \\ http://www.whatismyip.com/). En este caso dicha direcci\'on fue 24.232.212.124.
 \item Direcci\'on broadcast de la red (192.168.0.255)
 \item Direcci\'on 0.0.0.0
 \item Direcci\'on 255.255.255.255
 \item Direcciones inv\'alidas (por ejemplo: 123456789)
 \item Direcciones que no pertenecen a la red local (por ejemplo, 173.194.42.35)
\end{enumerate}

Al mismo tiempo que probamos estos casos con la funci\'on mencionada, utilizaremos tambi\'en la herramienta \emph{wireshark}, para poder ver con m\'as precisi\'on los mensajes que se env\'ian (como s\'olo nos interesan los paquetes ARP, utilizaremos un filtro que muestre solamente los mismos).

\subsection{Segunda Consigna: Capturando tr\'afico}

Para capturar los mensajes ARP nuevamente utilizaremos la herramienta \emph{scapy}: esta vez haremos uso de la funci\'on \emph{sniff}, filtrando solamente los mensajes ARP. Para hacer luego un an\'alisis de los datos obtenidos y poder comparar el comportamiento de distintas redes, tomamos los siguientes datos:

\begin{itemize}
 \item *** J., TE TOCA COMPLETAR!! Poner como era la red (ethernet creo, y no se que otro detalle) y por cuanto tiempo capturamos en cada lugar (en la oficina y en tu casa). Si podes, pone tambien cuantos nodos tenia cada red? No se, lo que te parezca!! ***
\end{itemize}

Utilizaremos adem\'as los archivos pcap provistos por la c\'atedra, con informaci\'on de una red de alto tr\'afico ARP, y otra de poco tr\'afico (\texttt{big\_arp.pcap} y \texttt{small\_arp.pcap}).\\

Una vez obtenidos los datos, analizaremos la entrop\'ia de la red. Para ello definimos una fuente de informaci\'on cuyos s\'imbolos consisten en pares \texttt{<ip\_fuente, ip\_destino>}\footnote{Consideramos los pares de manera ordenada, por lo cual no es lo mismo una combinaci\'on de IPs, que la misma combinaci\'on de manera invertida.}. Estos s\'imbolos son tomados de los mensajes ARP del tipo \emph{who-has} que env\'ia cada host\footnote{El sniffer programado captura ambos tipos de mensaje ARP; al calcular la entrop\'ia filtramos primero los mensajes \emph{who-has}.}. Recordemos que los mensajes \emph{is-at} se env\'ian de manera privada al nodo que pregunt\'o por esa IP, por lo tanto s\'olo podemos escuchar los del tipo \emph{who-has}, o los \emph{is-at} env\'iados a nuestra m\'aquina. Para evitar considerar que la m\'aquina que corre el sniffer es m\'as sigificativa de lo que realmente es, s\'olo observaremos los paquetes \emph{who-has}. Estos mensajes son los \'unicos que podemos ver en su totalidad, y son suficientes para conocer la interacci\'on entre hosts, si s\'olo observamos el protocolo ARP. \\

Cada s\'imbolo posible tiene una probabilidad que se relaciona con la importancia del nodo dentro de la red: los nodos \emph{significativos} (como por ejemplo el router) ser\'an aquellos que aparezcan con mayor frecuencia en el campo \texttt{ip\_destino}. La probabilidad de cada s\'imbolo se calcular\'a de manera emp\'irica, en base a la cantidad de apariciones sobre el total de s\'imbolos presentados, y dentro de los rangos de tiempo mencionados para la captura (esta es una de las razones por las que se tomaron datos durante varias horas *** O LA UNICA?). Podemos considerar que la fuente, definida de esta manera, es una fuente de memoria nula, ya que los s\'imbolos emitidos son estad\'sticamente independientes (*** CHAMUYO, NO ESTOY SEGURA DE ESTO, DE ULTIMA SAQUEMOSLO).\\

*** CONTAR SOBRE ENTROPIOMETRO, COMO ESTA HECHA LA FUNCION QUE USAMOS, LAS VENTANAS DE TIEMPO QUE TOMAMOS, ETC.\\
*** ACLARAMOS LO DE LA BASE DEL LOGARITMO, QUE TOMAMOS EN BASE 10 PORQUE TOTAL SE MULTIPLICA POR UNA CONSTANTE, O MEJOR EVITAMOS MENCIONARLO??



\subsection{Tercera Consigna: Gr\'aficos y An\'alisis}

Para caracterizar una red buscamos encontrar los nodos relevantes de la misma. Contamos para esto con los s\'imbolos mencionados en la secci\'on anterior (tuplas \texttt{<ip\_fuente, ip\_destino>}), y con el valor de la informaci\'on de cada nodo, asi como la entrop\'ia de la red\footnote{Para caracterizar la red, consideramos que la asociaci\'on entre una IP y una direcci\'on MAC no es relevante. Lo importante en este caso es la comunicaci\'on entre los distintos hosts, los cuales pueden representarse a trav\'es de su IP o de su direcci\'on MAC. Elegimos utilizar la IP, pero podr\'ia haber sido cualquiera de las dos opciones.}. A partir de estos datos podemos caracterizar cada una de las redes a estudiar observando la siguiente informaci\'on:

\begin{enumerate}
 \item Grafos dirigidos, en donde cada nodo representa una IP, y existe un eje desde el nodo A hasta el nodo B, si A hizo un pedido \emph{who-has} a B (es decir, si el s\'imbolo \texttt{<A, B>} apareci\'o durante la captura).
 \item Histograma relacionando cada IP, con la cantidad de veces que fue consultada (es decir, que apareci\'o en el campo \texttt{ip\_destino}).
 \item Cantidad de informaci\'on de cada IP. En este caso puede compararse la informaci\'on con la entrop\'ia total, y sacar conclusiones a partir de esto de cu\'ales son los nodos significativos.
 \item *** J: FALTA MENCIONAR TUS HISTOGRAMAS, QUE NO LOS ENTIENDO :)
 \item Entrop\'ia de la fuente en funci\'on de un rango horario. En este caso no estamos observando tanto los nodos significativos, sino c\'omo distintos tipos de redes pueden variar la cantidad de informaci\'on presente seg\'un el momento del d\'ia, y el tipo de red que sea -hogare\~na, oficina, etc. Este gr\'afico se relaciona a su vez con un histograma, en el que observaremos la cantidad de paquetes transferidos en funci\'on de la hora del d\'ia. *** PARA ESTO ULTIMIO FALTA JUSTIFICAR UN POCO MAS POR QUE LO ESTAMOS TOMANDO, Y QUE CONCLUSION VAMOS A PODER SACAR A PARTIR DE ESTO.
 \item *** LO DE LA ENTROPIA MEDIDA CADA HORA, VA SI ALGUIEN PUEDE EXPLICAR QUE QUIERE DECIR EL GRAFICO! (entropiaXHoraOficina)
\end{enumerate}

*** VER SI FALTO ALGO!!

Hay que aclarar que el gr\'afico (5) no fue realizados para los pcap provistos por la c\'atedra, ya que los mismos no contaban con un timestamp.


