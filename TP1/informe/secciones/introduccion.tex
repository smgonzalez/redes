\section{Introducci\'on}

% LES COPIO LO QUE TENGO ANOTADO EN EL CUADERNO:
% Hablar de Scapy, ARP, cosas que usamos (definiciones, etc.)

El presente trabajo pr\'actico tiene como objetivo estudiar el comportamiento de una red a nivel de enlace (analizando principalmente el v\'inculo de este nivel con la capa superior). Para ello vamos a observar el funcionamiento del protocolo ARP dentro de una determinada red. A su vez, relacionaremos las redes en esta capa con los conceptos aprendidos dentro del campo de la teor\'ia de la informaci\'on: estudiando la entrop\'ia de una red local (tomando como fuente de informaci\'on el modelo explicado m\'as adelante) buscamos caracterizar el funcionamiento de la misma, centr\'andonos en cuales son los nodos \emph{distinguidos} en ella.\\

ARP (\emph{Address Translation Protocol}) es el protocolo encargado de traducir direcciones IP a sus correspondientes direcciones MAC (es decir, las direcciones a nivel de enlace). El mismo permite que cada host, a trav\'es de mensajes \emph{broadcast} en los que se pregunta qui\'en tiene determinada IP, construya una tabla de asociaciones IP-MAC\footnote{Esta tabla se actualiza peri\'odicamente, ya que los mapeos pueden cambiar con el tiempo.}. Un frame ARP contiene, entre otros, los siguientes campos:

\begin{itemize}
 \item 
\end{itemize}
