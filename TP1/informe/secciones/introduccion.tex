\section{Introducci\'on}

% LES COPIO LO QUE TENGO ANOTADO EN EL CUADERNO:
% Hablar de Scapy, ARP, cosas que usamos (definiciones, etc.)

El presente trabajo pr\'actico tiene como objetivo estudiar el comportamiento de una red a nivel de enlace (analizando principalmente el v\'inculo de este nivel con la capa superior). Para ello vamos a observar el funcionamiento del protocolo ARP dentro de una determinada red. A su vez, relacionaremos las redes en esta capa con los conceptos aprendidos dentro del campo de la teor\'ia de la informaci\'on: estudiando la entrop\'ia de una red local (tomando como fuente de informaci\'on el modelo explicado m\'as adelante) buscamos caracterizar el funcionamiento de la misma, centr\'andonos en cuales son los nodos \emph{distinguidos} en ella.\\

ARP (\emph{Address Translation Protocol}) es el protocolo encargado de traducir direcciones IP a sus correspondientes direcciones de nivel de enlace (direcciones MAC *** MAC SON SOLO PARA ETHERNET???). El mismo permite que cada host, a trav\'es del env\'io de mensajes \emph{broadcast} dentro de la red local, construya una tabla de asociaciones IP-MAC, que le permitir\'an luego enviar mensajes de manera directa a las m\'aquinas que se encuentran dentro de su misma red. Para construir asociaciones IP-MAC, se env\'ia un paquete ARP consistente, entre otros, en los siguientes campos:

\begin{itemize}
 \item Tipo de mensaje (\emph{who-is} o \emph{is-at})
 \item IP del host que env\'ia el mensaje
 \item MAC del host que env\'ia el mensaje
 \item IP del host destino
 \item MAC del host destino
\end{itemize}

Si la m\'aquina A quiere mandar un mensaje a B, conociendo solamente su direcci\'on IP, se env\'ia de manera \emph{broadcast} un mensaje ARP del tipo \emph{who-is}, con la IP que se quiere conocer dentro del campo IP-destino. El host cuya IP se corresponda con el mensaje, contestar\'a entonces con un ARP del tipo \emph{is-at}, indicando su direcci\'on a nivel de enlace. Este \'ultimo mensaje se env\'ia solamente a la IP que desea conocer la MAC, por lo cual s\'olo puede ser escuchado si por la m\'aquina que pregunt\'o.\\

El protocolo ARP permite mapear otros tipos de direcciones adem\'as de Ip y Ethernet. En este trabajo, sin embargo, nos concentraremos solamente en este \'ultimo caso
