\begin{abstract}
%\textbf{Resumen:} 
En el presente trabajo estudiaremos el comportamiento de una red a nivel de enlace, enfoc\'andonos en la relaci\'on entre esta capa y la capa superior. Para esto analizaremos los paquetes del protocolo ARP enviados a trav\'es de una red local, escuchando pasivamente durante varias horas la red seleccionada. Esta informaci\'on va a ser utilizada luego para ver, a trav\'es de grafos e histogramas, cu\'ales son los nodos significativos dentro de la misma. A su vez, utilizaremos los conceptos de entrop\'ia e informaci\'on para caracterizar las redes estudiadas, tomando los s\'imbolos de la fuente de alguna manera conveniente. Tanto para el env\'io de paquetes ARP, como para la captura de datos de la red, utilizaremos las herramientas \emph{Wireshark} y \emph{Scapy}.\\

Los experimentos ser\'an realizados en cuatro tipos de redes distintos. Los resultados obtenidos son los esperados seg\'un el uso que se le da a cada una de estas redes.\\

\textbf{Palabras clave:} ARP, Teoría de la Información, Entropía, Scapy, Wireshark, Nivel de Enlace, IP, MAC, LAN.
\end{abstract}
