
\section{Introducci\'on}

El presente trabajo pr\'actico tiene como objetivo explorar el nivel de red, observando las rutas que siguen los paquetes IPs al viajar largas distancias. Para ello implementaremos dos herramientas de análisis: \texttt{ping} y \texttt{traceroute},  basándonos en el protocolo ICMP (Internet Control Message Protocol).\\

Dicho protocolo se utiliza para control y notificaci\'on de errores en el envi\'o de paquetes IP, por lo que corre por sobre este. Las herramientas \texttt{ping} y \texttt{traceroute} que se encuentran en los sistemas operativos más utilizados se basan en este mismo protocolo.\\

De los varios tipos de paquetes ICMP existentes, para la implementaci\'on de las herramientas nos focalizaremos en tres de ellos, cuyo uso comentamos a continuación:

\begin{itemize}
 \item Echo-Request \\ 
 Solicita la respuesta de un host. Por ejemplo para saber si un host está disponible, puede enviársele un paquete ICMP de este tipo.
 \item Echo-Reply \\
 Informa la respuesta de un host. Al recibirse un paquete de tipo {\bf Echo-Request}, se le envía al remitente uno de este tipo, indicando que el host está efectivamente activo y disponible.
 \item Time Exceeded \\
 Existe una propiedad de los paquetes ICMP denominada {\bf Time To Live} (TTL) que limita el número de {\bf hops} que puede atravesar: cuando se construye el paquete se le asigna un valor numérico, que es decrementado con cada hop de routing; si un router recibe un paquete de tipo {\bf Echo-Request} con TTL = 0, envía un paquete ICMP de tipo {\bf Time Exceeded} al remitente del paquete. En esta propiedad del protocolo basaremos nuestra implementación de la herramienta Traceroute.
\end{itemize}

%Cuando un host A quiere saber si un host B esta disponible, lo que hace es mandar un paquete ICMP (sobre uno IP) de tipo {\bf Echo-Request}. Si el servidor B esta disponible efectivamente, le env\'ia al host un paquete ICMP de tipo {\bf Echo-Reply}.

%Esto por ejemplo se puede utilizar para aproximar una ruta posible que sigue un paquete, enviando paquetes de tipo Echo-Request aumentando linealmente el TTL, y viendo que IPs fuente son las que nos mandan las respuestas de Time Exceeded, hasta que se recibe un Echo-Reply.

%En base a este algoritmo se implementara la herramienta {\bf Traceroute}. 
Utilizaremos estas herramientas (en nuestras implementaciones, además de las ya implementadas en sistemas Unix) para realizar exploraciones, mediciones y an\'alisis de los enlaces de Internet, haciendo énfasis en enlaces transoceánicos. Para identificar estos últimos, utilizaremos además herramientas como las que se encuentran en \url{http://www.geoiptool.com/es/} o \url{http://geoip.flagfox.net/} para poder localizar (aproximadamente) el punto geogr\'afico de un router mediante su dirección IP pública.

