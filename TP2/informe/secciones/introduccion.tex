\section{Introducci\'on}

El presente trabajo pr\'actico consiste en analizar las rutas que siguen los paquetes IPs al viajar largas distancias. Para ello implementaremos 2 herramientas basadas en el protocolo ICMP:
Ping y Traceroute.

El protocolo ICMP (Internet Control Message Protocol) suele utilizarse como un protocolo de control entre y notificacion de errores en el envio de paquetes IP, por lo que corre debajo de este mismo. Las herramientas Ping y Traceroute que se encuentran en los Sistemas Operativos mas utilizados utilizan este mismo protocolo.

Existen muchos tipos de paquetes ICMP que pueden mandarse, pero nosotros para la implementacion de las herramientas solo nos focalizaremos en 3:

\begin{itemize}
 \item Echo-Request
 \item Echo-Reply
 \item Time Exceeded
\end{itemize}

Cuando un host A quiere saber si un host B esta disponible, lo que hace es mandar un paquete ICMP (sobre uno IP) de tipo {\bf Echo-Request}. Si el servidor B esta disponible efectivamente, le envia al host un paquete ICMP de tipo {\bf Echo-Reply}.

Existe una propiedad de los paquetes ICMP denominada {\bf Time To Live} (o TTL) el cual indica el numero maximo de {\bf hops} que puede realizar el mismo. Cuando un router resive un paquete de tipo Echo-Request el cual tiene como TTL = 0, este manda un paquete ICMP de tipo {\bf Time Exceeded} al servidor origen Esto por ejemplo se puede utilizar para aproximar una ruta posible que sigue un paquete, enviando paquetse de tipo Echo-Request aumentando linealmente el TTL, y viendo que IPs fuente son las que nos mandan las respuestas de Time Exceeded, hasta que se recibe un Echo-Reply.

En base a este algoritmo se implementara la herramienta {\bf Traceroute}. Luego se utilizaran tanto estas herramientas como las ya implementadas en sistemas Unix para realizar mediciones y analisis en base a las rutas y los enlaces transoceanicos que se encuentren.

Para determinar si un enlace es Transoceanico lo que se hara sera es utilizar herramientas como las que se encuentran en \url{http://www.geoiptool.com/es/} o \url{http://geoip.flagfox.net/} Para poder localicar el punto geografico de un router mediante su IP publica.

%% ACA ME FALTA UN TOQUE, COMITIE PORQUE ME FUI A COMER :P
