\section{M\'etodos}
 
 \subsection{Implementaci\'on de \texttt{ping} y \texttt{traceroute}}
 
 Para implementar \texttt{ping} debemos enviar un paquete ICMP del tipo \texttt{echo-request} a la direcci\'on deseada, y esperar de respuesta el correspondiente paquete ICMP del tipo \texttt{echo-reply}.\\
 
 En nuestro caso, utilizamos la herramienta \texttt{scapy} para generar, enviar y esperar la respuesta del paquete. Inicializamos un paquete ICMP sobre un paquete IP, de la siguiente manera:\\
 
 \begin{center}
    \texttt{icmp = IP(dst=host, ttl=p\_ttl) / ICMP()}
 \end{center}
 
 Por defecto, el paquete ICMP se inicializa con tipo \texttt{echo-request}, por lo tanto basta con enviar el paquete creado y esperar la respuesta. Esto \'ultimo se consigue utilizando la funci\'on \texttt{sr1}, que se encarga de ambas acciones:
 
 \begin{center}
    \texttt{ans = sr1(icmp)}
 \end{center}
 
 Como resultado de esta funci\'on  se obtiene el paquete ICMP  \texttt{echo-reply} devuelto por el host al recibir el \texttt{echo-request} (si se produjo la respuesta), con la direcci\'on IP de donde proviene, entre otros datos.\\
 
 La implementaci\'on de \texttt{ping} permite definir, adem\'as del host al que se env\'ia el paquete, el ttl que aparecer\'a en la cabecera IP.\\
  
% De esta forma se puede comprobar de una forma muy sencilla si un host esta disponible o no. De no estarlo, el paquete ICMP se perderia y nunca seria respondido, por lo que se deduce que el paquete se perdio o llego a un host que nose encontraba en linea.
 
 Para la implementaci\'on de \texttt{traceroute} también utilizamos las funciones provistas por \texttt{scapy}. Este m\'etodo recibe como par\'ametros de entrada la direcci\'on destino, la cantidad m\'axima de routers a recorrer (\texttt{ttl\_max}), y la cantidad de paquetes a enviar para sondear cada nodo (\texttt{pkgs\_per\_ttl}). Como dijimos, utilizaremos una aproximación mediante ICMP para rastrear la ruta, de la siguiente manera: supongamos que comenzamos con un paquete cuyo \texttt{ttl} es igual a 1; el router que reciba este paquete lo descartará, devolviendo a la direcci\'on fuente del mensaje un paquete ICMP indicando lo ocurrido, mientras que si enviamos uno con $\texttt{ttl} = 2$, el mensaje de error será enviado por el segundo router que alcanza. M\'as precisamente, sabemos que al mandar un paquete con \texttt{ttl = n} recibiremos (probablemente) como respuesta un mensaje ICMP del tipo \texttt{time-exceeded}, con c\'odigo \texttt{ttl-zero-during-transit}, y la direcci\'on IP del n-ésimo router 
que recibió el paquete (y decrement\'o el \texttt{ttl} a cero). Así, aumentando sucesivamente el \texttt{ttl}  podemos inferir la ruta que sigue hasta llegar a destino, terminando cuando se llega al m\'aximo de hops permitidos (definido por \texttt{ttl\_max}), o al recibir un paquete de tipo \texttt{echo-reply}, indicando que se alcanzó el destino y éste respondi\'o correctamente.\\

La implementaci\'on realizada permite conocer tambi\'en el RTT estimado hacia cada nodo, midiendo el tiempo de ejecuci\'on de la funci\'on \texttt{sr1}. Debido a que este tiempo puede variar considerablemente seg\'un el tr\'afico que haya en el momento, para tomar un valor representativo es conveniente enviar varios paquetes con el mismo \texttt{ttl} y calcular un RTT promedio.\\
 
 Hay que tener en cuenta que no todos los nodos responderán el \texttt{ping}: muchos routers ignoran estos paquetes por cuestiones de seguridad; o dado que ICMP es generalmente el protocolo con menor prioridad, un router ocupado puede elegir desechar este tipo de mensaje (mientras que otros ser\'ian recibidos correctamente). Si por alguna de estas razones no se recibe respuesta después de un determinado tiempo, se procede enviando el siguiente paquete. Por esto no esperamos obtener una ruta completa, sino solamente algunos de los routers por donde pasa.\\
 
 Otro problema de esta implementaci\'on es que no podemos asegurar que, al enviar cada mensaje con un determinado \texttt{ttl}, se siga la ruta encontrada hasta el momento: por razones de congesti\'on (entre otras), el routeo IP permite alterar las rutas de los paquetes, aún para aquéllos con mismo origen y destino (incluso para cada paquete enviado). Por esta raz\'on, y por lo mencionado anteriormente respecto de la p\'erdida de paquetes, el camino que encuentre \texttt{traceroute} va a ser un camino \emph{aproximado}. \\
 
 % Mencionar en resultados que pensamos que siempre sigue la misma ruta, porque los resultados muestran siempre lo mismo? (chequear). Por lo menos el enlace es siempre el mismo
 
 \subsection{Exploraci\'on y Medición}
 
 % CONTAR COMO LOCALIZAMOS CADA IP, COMO TOMAMOS EL RTT MINIMO, ETC. DESPUES EN RESULTADOS PONEMOS LOS GRAFIQUITOS Y LOS ANALIZAMOS
 Utilizando las funciones mencionadas en la secci\'on anterior buscamos encontrar enlaces que conecten distintos continentes. Para ello seleccionamos tres direcciones web correspondientes a pa\'ises de Europa y Africa, y corrimos \texttt{traceroute} hacia ellos. Inspeccionando las IPs obtenidas con herramientas de localización geográfica, podemos detectar entre cuáles de ellas existe un enlace transoceánico\footnote{Como dijimos, esta ruta es una aproximación que utilizamos para acotar la ubicación de los extremos de los enlaces, aunque no nos permite conocer que efectivamente exista un enlace directo entre ellos.}. Las p\'aginas utilizadas fueron las siguientes:
 
 \begin{itemize}
  \item Inglaterra - Universidad de Oxford: \url{www.ox.ac.uk}
  \item Finlandia - Ciudad de Helsinki: \url{www.helsinki.fi}
  \item Sud\'africa - Universidad de Pretoria: \url{web.up.ac.za}
 \end{itemize}
 
 El enlace transoce\'anico se encuentra analizando el resultado de \texttt{traceroute} hacia cada una de estas p\'aginas. Dadas las IPs por donde pasa el paquete, podemos utilizar herramientas de geolocalizaci\'on para trazar la ruta y encontrar el momento en que se cruza hacia otro continente. Debido a que la ruta encontrada por \texttt{traceroute} no es exacta, tampoco lo ser\'an los enlaces encontrados; sin embargo pueden servirnos para darnos una idea aproximada de por d\'onde se transmiten los paquetes. Para conocer la ubicaci\'on de una IP utilizamos las siguientes p\'aginas web:
 
\begin{itemize}
 \item IP Location (\url{www.iplocation.net})\\
 Esta web indica la localizaci\'on de una IP seg\'un el resultado de otras 3 p\'aginas: IP2Location, IPLigence y IP Adress Lab. Permite comparar r\'apidamente el resultado de tres bases de datos distintas (las cuales muchas veces var\'ian considerablemente).
 \item Geo IP Tool (\url{www.geoiptool.com})
 \item Geotool (\url{geoip.flagfox.net})
\end{itemize}

 Todas las herramientas de geolocalizaci\'on indican el lugar ge\'ografico de una IP bas\'andose en bases de datos, las cuales pueden ser incorrectas o encontrarse desactualizadas. Por esta raz\'on, para saber la ubicaci\'on de una IP consultamos todas las webs mencionadas, seleccionando (si no coincid\'ian en las respuestas) la ubicaci\'on m\'as coherente con respecto a la ruta que se ven\'ia siguiendo. Nuevamente, hay que tener en cuenta que la ruta finalmente encontrada es una ruta aproximada; ya no s\'olo porque la manera de encontrar las IPs no es exacta, sino tambi\'en porque la informaci\'on respecto de la ubicaci\'on de las mismas no es del todo confiable. \\
 
 
 Para conocer el comportamiento de la red en distintos momentos del d\'ia corrimos \texttt{traceroute}, para cada host mencionado, durante $24$ horas, registrando el RTT promedio\footnote{Promediado sobre $100$ paquetes env\'iados por cada \texttt{ttl}, es decir, por cada nodo de la ruta. En caso de no recibir respuesta de alguno de estos paquetes no se incluye su valor de RTT dentro del promedio.} para cada uno de los extremos del enlace determinado. Para ahorrar paquetes y acelerar el proceso, acotamos el rango de \texttt{ttl} de los paquetes, estableciendo un m\'inimo y un m\'aximo cercanos a la ubicaci\'on (en n\'umero de saltos) del enlace encontrado. Si bien teóricamente podríamos haber enviado \texttt{ping}s solamente a los extremos del enlace elegido, observamos que muchas veces estos paquetes no eran respondidos, aunque sí obteníamos respuesta en caso de \emph{time-exceeded}. Consideramos como RTT del enlace la diferencia entre el RTT promedio hacia una punta, y el RTT promedio hacia la otra; la 
latencia es entonces la mitad de este valor.\\

Es interesante conocer tambi\'en el valor te\'orico del RTT: sabiendo la distancia del enlace encontrado y el tipo de medio f\'isico, cu\'al es el m\'inimo valor de RTT esperado. Para esto asumimos que el enlace es de fibra \'optica, con una velocidad de propagaci\'on de la se\~nal de $2x10^5$ $km/s$. Para conocer la distancia del enlace utilizamos un calculador de distancias a partir de las coordenadas geogr\'aficas (las cuales obtuvimos anteriormente mediante la IP)\footnote{}. Debido a que la localizaci\'on no es exacta, y las asunciones sobre el medio f\'isico del enlace podr\'ian no ser ciertas, puede ser que el valor te\'orico calculado no sea menor a los RTTs encontrados emp\'iricamente.\\

Por \'ultimo, vamos a explorar los enlaces hayados enviando sondas de \texttt{traceroute} a otras p\'aginas del continente (distintas a las mencionadas arriba). Queremos observar el uso de estos enlaces hacia distintas ubicaciones. *** ESTO FALTA!!!!

% RTTs mal: el enlace encontrado, y la distancia entre los nodos, podr\'ia ser otra.\\ El enlace tiene nodos en el medio con tiempos de encolamiento, no es directo.
 
 %Nuevamente, no podemos asegurar que todos nuestros paquetes hayan seguido la misma ruta hasta cada uno de los extremos de interés, razón por la cual realizamos el experimento para un $n$ relativamente grande, para suavizar los desvíos que puedan sufrir algunos paquetes..
 %PARA QUÉ N?!?!?!
