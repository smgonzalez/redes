\section{M\'etodos}
 
 \subsection{Implementaci\'on de \texttt{ping} y \texttt{traceroute}}
 
 Para implementar \texttt{ping} debemos enviar un paquete ICMP del tipo \texttt{echo-request} a la direcci\'on deseada, y esperar de respuesta el correspondiente paquete ICMP del tipo \texttt{echo-reply}. Esto se puede conseguir utilizando la herramienta \texttt{scapy}, inicializando un paquete ICMP sobre un paquete IP, de la siguiente manera:\\
 
 \begin{center}
    \texttt{icmp = IP(dst=host, ttl=p\_ttl) / ICMP()}
 \end{center}
 
 La funci\'on permite definir el host al que se quiere enviar el paquete IP y el TTL. Por defecto, el paquete ICMP se inicializa con tipo \texttt{echo-request}, por lo tanto basta con enviar el paquete creado y esperar la respuesta. Esto \'ultimo se consigue utilizando la funci\'on \texttt{sr1}, la cual se encarga de ambas acciones:
 
 \begin{center}
    \texttt{ans = sr1(icmp)}
 \end{center}
 
 Como resultado de esta funci\'on, si es que se recibe respuesta, se obtiene el paquete ICMP devuelto por el host al recibir el \texttt{echo-request}, con la direcci\'on IP de donde proviene, entre otros datos.\\
 *** ALGUNA ESPECIE DE REFLEXION DE QUE NOS CONTESTARON Y ENTONCES ESTAN VIVOS? DE CUANDO CONTESTAN Y CUANDO NO?\\
 
 Para la implementaci\'on de \texttt{traceroute} nuevamente vamos a utilizar las funciones provistas por \texttt{scapy}. Este m\'etodo recibe como par\'ametros de entrada la direcci\'on destino, la cantidad m\'axima de gateways que se va a recorrer (\texttt{ttl\_max}), y la cantidad de paquetes a enviar a cada nodo que se encuentra en la ruta (\texttt{pkgs\_per\_ttl}).\\
 
 La ruta que sigue un paquete puede ser encontrada variando el \texttt{ttl} dentro de la cabecera IP. Si comenzamos con un \texttt{ttl} igual a 1, el gateway que reciba ese paquete lo va a descartar, devolviendo a la direcci\'on fuente del mensaje un paquete ICMP indicando lo ocurrido. M\'as precisamente, sabemos que al mandar un paquete con \texttt{ttl = 1} recibiremos (probablemente) como respuesta un mensaje ICMP del tipo \texttt{time-exceeded}, con c\'odigo \texttt{ttl-zero-during-transit}, y la direcci\'on IP del gateway que decrement\'o el \texttt{ttl} a cero. Esto nos permite saber la direcc\'on IP del primer nodo al que se env\'ia el paquete. Aumentando sucesivamente el \texttt{ttl} (\texttt{ttl=2}, \texttt{ttl=3}, etc) podemos descubrir la ruta que sigue hasta llegar a destino.\\
 
 El paquete a enviar es simplemente un \texttt{ping}, el cual se manda una determinada cantidad de veces (seg\'un la variable \texttt{pkgs\_per\_ttl}) a la direcci\'on destino, para luego incrementar el ttl y repetir, llegando asi al siguiente nodo dentro de la ruta.\\
 
 El algoritmo frena cuando se llega a un m\'aximo de hops permitidos (definido por \texttt{ttl\_max}), o cuando se recibe un paquete del tipo \texttt{echo-reply}, lo cual indica que se lleg\'o a destino y el mismo respondi\'o correctamente.\\
 
 La implementaci\'on realizada permite conocer tambi\'en el RTT estimado hacia cada nodo. Esto se consigue midiendo el tiempo de ejecuci\'on de la funci\'on \texttt{sr1}. Debido a que este tiempo puede variar considerablemente seg\'un el tr\'afico que haya en el momento, es conveniente enviar una cantidad de paquetes relativamente grande por cada ttl, y tomar el RTT promedio.\\
 
 Hay que tener en cuenta que no todos los nodos van a responder el \texttt{ping}: muchos de ellos ignoran estos paquetes por cuestiones de seguridad. Adem\'as, ICMP es generalmente el protocolo con menor prioridad, con lo cual un router ocupado puede elegir desechar este mensaje, sin que esto signifique que desecha todos los mensajes enviados a \'el. Si por alguna de estas razones no se recibe respuesta se procede a enviar el siguiente paquete. Puede ser que ocurra, y de hecho ocurre en todas las implementaciones de \texttt{traceroute}, que algunos de los nodos dentro de la ruta generen una p\'erdida de paquetes (para todos los paquetes enviados a \'el), con lo cual no vamos a obtener una ruta completa, sino solamente algunos de los gateways por donde pasa.\\
 
 Otro problema de esta implementaci\'on es que no podemos asegurar que, al enviar un mensaje con un determinado \texttt{ttl} (menor a lo necesario para llegar a destino) se siga la ruta encontrada hasta el momento. Es decir, podr\'ia pasar que el paquete enviado hasta ese nodo vaya por una ruta diferente a la calculada hasta el momento, probablemente por razones de congesti\'on. Por esta raz\'on, y por lo mencionado anteriormente respecto de la p\'erdida de paquetes, el camino que encuentre \texttt{traceroute} va a ser un camino aproximado. \\
 
 % Mencionar en resultados que pensamos que siempre sigue la misma ruta, porque los resultados muestran siempre lo mismo? (chequear). Por lo menos el enlace es siempre el mismo
 
 \subsection{Exploraci\'on}
 
 % CONTAR QUE ENLACES ELEGIMOS, POR QUE, COMO LOCALIZAMOS CADA IP, COMO TOMAMOS EL RTT MINIMO, ETC. DESPUES EN RESULTADOS PONEMOS LOS GRAFIQUITOS Y LOS ANALIZAMOS
Las funciones mencionadas en la secci\'on anterior van a ser utilizadas para encontrar enlaces que conecten distintos continentes. Para ello seleccionamos tres direcciones correspondientes a pa\'ises de Europa y Africa, y corrimos \texttt{traceroute} con ellos. Dadas las IPs por donde pasa el paquete, podemos utilizar herramientas de geolocalizaci\'on para trazar la ruta y encontrar el momento en que se cruza hacia otro continente.\\

Como explicamos en la secci\'on anterior, la ruta encontrada es solamente una ruta \emph{aproximada}, por lo tanto no esperamos encontrar el enlace transatl\'antico exacto. Sin embargo los nodos obtenidos pueden darnos una idea de por d\'onde se encuentra el mismo. Para conocer la ubicaci\'on de una IP utilizamos las siguientes p\'aginas: 