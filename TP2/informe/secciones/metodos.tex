\section{M\'etodos}
 
 \subsection{Implementaci\'on de \texttt{ping} y \texttt{traceroute}}
 
 Para implementar \texttt{ping} debemos enviar un paquete ICMP del tipo \texttt{echo-request} a la direcci\'on deseada, y esperar de respuesta el correspondiente paquete ICMP del tipo \texttt{echo-reply}.
 
 En nuestro caso, utilizamos la herramienta \texttt{scapy} para generar, enviar y esperar la respuesta del paquete. Inicializando un paquete ICMP sobre un paquete IP, de la siguiente manera:\\
 
 \begin{center}
    \texttt{icmp = IP(dst=host, ttl=p\_ttl) / ICMP()}
 \end{center}
 
 La funci\'on permite, además, definir el host al que se quiere enviar el paquete IP y el TTL. Por defecto, el paquete ICMP se inicializa con tipo \texttt{echo-request}, por lo tanto basta con enviar el paquete creado y esperar la respuesta. Esto \'ultimo se consigue utilizando la funci\'on \texttt{sr1} que se encarga de ambas acciones:
 
 \begin{center}
    \texttt{ans = sr1(icmp)}
 \end{center}
 
 Como resultado de esta funci\'on,  se obtiene el paquete ICMP  \texttt{echo-reply} devuelto por el host al recibir el \texttt{echo-request} (si se produjo al respuesta), con la direcci\'on IP de donde proviene, entre otros datos.\\
  
% De esta forma se puede comprobar de una forma muy sencilla si un host esta disponible o no. De no estarlo, el paquete ICMP se perderia y nunca seria respondido, por lo que se deduce que el paquete se perdio o llego a un host que nose encontraba en linea.
 
 Para la implementaci\'on de \texttt{traceroute} también utilizamos las funciones provistas por \texttt{scapy}. Este m\'etodo recibe como par\'ametros de entrada la direcci\'on destino, la cantidad m\'axima de routers a recorrer (\texttt{ttl\_max}), y la cantidad de paquetes a enviar para sondear cada nodo(\texttt{pkgs\_per\_ttl}). Como dijimos, utilizaremos una aproximación mediante ICMP para rastrear la ruta, de la siguiente manera: supongamos que comenzamos con un paquete cuyo \texttt{ttl} es igual a 1; el router que reciba este paquete lo descartará, devolviendo a la direcci\'on fuente del mensaje un paquete ICMP indicando lo ocurrido, mientras que si enviamos uno con $\texttt{ttl} = 2$, el mensaje de error será enviado por el segundo router que alcanza. M\'as precisamente, sabemos que al mandar un paquete con \texttt{ttl = n} recibiremos (probablemente) como respuesta un mensaje ICMP del tipo \texttt{time-exceeded}, con c\'odigo \texttt{ttl-zero-during-transit}, y la direcci\'on IP del n-ésimo router que recibió el paquete (y decrement\'o el \texttt{ttl} a cero). Así, aumentando sucesivamente el \texttt{ttl}  podemos inferir la ruta que sigue hasta llegar a destino, terminando cuando se llega al m\'aximo de hops permitidos (definido por \texttt{ttl\_max}), o al recibir un paquete de tipo \texttt{echo-reply}, indicando que se alcanzó el destino y éste respondi\'o correctamente. Además, la implementaci\'on realizada permite conocer tambi\'en el RTT estimado hacia cada nodo, midiendo el tiempo de ejecuci\'on de la funci\'on \texttt{sr1}. Debido a que este tiempo puede variar considerablemente seg\'un el tr\'afico que haya en el momento, para tomar un valor representativo es conveniente enviar varios paquetes con el mismo \texttt{ttl}, calcular un RTT promedio.\\
 
 También hay que tener en cuenta que no todos los nodos responderán el \texttt{ping}: muchos routers ignoran estos paquetes por cuestiones de seguridad, o dado que ICMP es generalmente el protocolo con menor prioridad, un router ocupado puede elegir deshechar este mensaje, sin que esto signifique que deshecha todos los mensajes enviados a \'el. Si por alguna de estas razones no se recibe respuesta después de un determinado tiempo, se procede enviando el siguiente paquete. Por esto no esperamos obtener una ruta completa, sino solamente algunos de los routers por donde pasa.\\
 
 Otro problema de esta implementaci\'on es que no podemos asegurar que, al enviar cada mensaje se siga la ruta encontrada hasta el momento: por razones de congesti\'on (entre otras), el routeo IP permite alterar las rutas de los paquetes, aún para aquéllos con mismo origen y destino (incluso para cada paquete enviado). Por esta raz\'on, y por lo mencionado anteriormente respecto de la p\'erdida de paquetes, el camino que encuentre \texttt{traceroute} va a ser un camino aproximado. \\
 
 % Mencionar en resultados que pensamos que siempre sigue la misma ruta, porque los resultados muestran siempre lo mismo? (chequear). Por lo menos el enlace es siempre el mismo
 
 \subsection{Exploraci\'on y Medición}
 
 % CONTAR QUE ENLACES ELEGIMOS, POR QUE, COMO LOCALIZAMOS CADA IP, COMO TOMAMOS EL RTT MINIMO, ETC. DESPUES EN RESULTADOS PONEMOS LOS GRAFIQUITOS Y LOS ANALIZAMOS
 Utilizamos las funciones mencionadas en la secci\'on anterior para encontrar enlaces que conecten distintos continentes. Para ello seleccionamos tres direcciones web correspondientes a pa\'ises de Europa y Africa, y corrimos \texttt{traceroute} hacia ellos. Inspeccionando las IPs obtenidas con las herramientas de localización geográfica, podemos detectar entre cuáles de ellas existe un enlace transoceánico\footnote{Como dijimos, esta ruta es una aproximación que utilizamos para acotar la ubicación de los extremos de los enlaces, aunque no nos permite conocer efectivamente que exista uno entre ellos.}. %poner qué pagina usamos, por lo menos
 
 Una vez detectado el enlace, enviamos sistemáticamente $n$ \texttt{traceroute}s al mismo destino, detectando en cada uno la ubicación del enlace elegido y registrando el RTT para cada uno de los extremos determinados. Para ahorrar paquetes y acelerar el proceso, acotamos el rango de \texttt{ttl} de los paquetes enviados para que alcancen el enlace buscado, con un mínimo margen. Si bien teóricamente podríamos haber enviado \texttt{ping}s a cada uno de los extremos, observamos que muchas veces estos paquetes no eran respondidos, aunque sí obeníamos respuesta en caso de \emph{time-exceeded}s. Nuevamente, no podemos asegurar que todos nuestros paquetes hayan seguido la misma ruta hasta cada uno de los extremos de interés, razón por la cual realizamos el experimento para un $n$ relativamente grande, para suavizar los desvíos que puedan sufrir algunos paquetes. Finalmente, consideramos como latencia del enlace la mitad de la diferencia entre los RTT promedio de cada uno de sus extremos.
 %PARA QUÉ N?!?!?!
